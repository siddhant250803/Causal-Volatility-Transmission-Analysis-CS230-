\documentclass{article}
\usepackage[margin=1in]{geometry}
\usepackage{amsmath}
\usepackage{parskip}

\title{Executive Summary: Causal Volatility Transmission Analysis}
\author{}
\date{}

\begin{document}

\maketitle

\section*{Executive Summary}

This project develops a deep learning framework for discovering and analyzing causal relationships in high-frequency stock volatility transmission. Using attention-based neural networks, we model how volatility shocks propagate between stocks with heterogeneous time lags, enabling identification of leading indicators and contagion pathways in financial markets. The system processes 5-minute interval stock returns from 252,021 time steps (2004-2016) across up to 200 stocks, predicting realized volatility while simultaneously learning the causal graph structure.

Our methodology employs a novel \textit{Causal Attention Model} with learned temporal lags and gating mechanisms. The architecture consists of: (1) stock embeddings (\(d_{\text{model}} = 64\)) that encode historical returns over a 7-hour lookback window (84 intervals), (2) a differentiable lagged attention mechanism with soft temporal indexing that learns continuous lag values \(\tau_i \in [0, 360]\) minutes (6 hours) for each source stock \(i\), and (3) learnable causal gates \(g_i \in [0,1]\) that determine influence strength. The model is trained with MSE loss plus balanced regularization: \(\mathcal{L} = \mathcal{L}_{\text{MSE}} + \lambda \|g\|_2 + \beta \mathcal{L}_{\text{lag-div}} + \gamma \mathcal{L}_{\text{TV}} + \eta \mathcal{L}_{\text{IRM}}\), where terms enforce sparse gate selection (\(\lambda = 0.005\)), lag diversity (\(\beta = 0.01\)), attention smoothness (\(\gamma = 0.001\)), and cross-regime stability (\(\eta = 0.001\)). Relationships are filtered using relative thresholding (top 50\% of learned gates) and validated using classical Granger causality tests (F-tests with \(p < 0.05\)) to distinguish genuine temporal precedence from spurious correlations. The framework achieves \(R^2 > 0.99\) on validation sets while producing sparse, interpretable causal graphs showing which stocks influence each target and at what time delays, with applications in risk management, portfolio construction, and systemic risk monitoring.

\end{document}

